\documentclass[a4paper]{article}

\usepackage{hyperref}
\usepackage{listings}
\usepackage{xcolor}

\title{Boids Project}
\author{Your Name}
\date{\today}

\begin{document}

\maketitle

\section{Introduction}
This project is an implementation of the Boids algorithm for simulating the flocking behavior of birds. Boids is a simple model that describes the collective motion of a group of autonomous agents, such as birds or fish. This README provides an overview of the project and instructions for running it.

\section{Installation}
To run this project locally, follow these steps:

\begin{enumerate}
    \item Clone the repository:
    \begin{lstlisting}[language=bash, backgroundcolor=\color{lightgray}]
    git clone https://github.com/Greg301003/Boids.git
    \end{lstlisting}
    
    \item Change to the project directory:
    \begin{lstlisting}[language=bash, backgroundcolor=\color{lightgray}]
    cd boids-project
    \end{lstlisting}
    
    \item Install the required dependencies:
    \begin{lstlisting}[language=bash, backgroundcolor=\color{lightgray}]
    pip install -r requirements.txt
    \end{lstlisting}
\end{enumerate}

\section{Usage}
To run the Boids simulation, execute the following command:
\begin{lstlisting}[language=bash, backgroundcolor=\color{lightgray}]
python boids_simulation.py
\end{lstlisting}

\section{Contributing}
If you would like to contribute to this project, follow these steps:

\begin{enumerate}
    \item Fork the repository.
    \item Create a new branch for your feature or bug fix.
    \item Make your changes and commit them.
    \item Push your changes to your forked repository.
    \item Open a pull request in the original repository.
\end{enumerate}

\section{License}
This project is licensed under the MIT License. See the \texttt{LICENSE} file for more details.

\section{Contact}
If you have any questions or suggestions, feel free to contact me at your-email@example.com.

\end{document}